\section{Introducción}

Frente al rápido crecimiento de \emph{Internet} y su enorme infraestructura
subyacente, las rutas utilizadas para el tráfico de paquetes entre nodos, con su
gran número de factores que influyen en su naturaleza cambiante, resultan ser un
objeto de gran interés.

El estudio realizado se concentra en el uso del protocolo de control
\texttt{ICMP} para obtener los nodos involucrados en una ruta y la estimación de
saltos intercontinentales mediante la medición de los \texttt{RTT} de los
paquetes utilizados. En el mismo, se experimentó utilizando como destinos los
sitios web de tres universidades del mundo para así tener la posibilidad de
obtener varios saltos intercontinentales y rutas de distintas longitudes.

Para cada ruta analizada se presenta un mapa con la geolocalización de las
direcciones \texttt{IP} pertenecientes a la traza junto a un desarrollo sobre el
nivel de efectividad de los métodos aplicados tanto para estimar la estructura
de los caminos como la detección de saltos intercontinentales.
