\section{Conclusiones}

Del trabajo realizado pueden extraerse conclusiones interesantes acerca del
poder de las herramientas utilizadas a la hora de analizar la topología y el
funcionamiento de las rutas en \emph{Internet}.

Es posible notar cómo, para rutas largas, la capacidad de detección de los
saltos intercontinentales parece decaer. Esto es especialmente notorio en el
caso del escenario A, correspondiente a la ruta más extensa de las tres, donde
solo pudo detectarse correctamente uno de tres saltos intercontinentales. Como
ya se discutió en el análisis de esta ruta, esto puede deberse a que las
rutas de tal longitud resultan poco estables. Durante el tiempo que demora
en realizarse el análisis de \emph{traceroute}, es posible que partes de la
ruta se modifiquen, cambiando, por ejemplo, el número de \emph{hop}
correspondiente a un salto intercontinental. De suceder esto, se produciría
un claro efecto negativo en la posibilidad de identificar dicho salto.

Algo para destacar es que, en la mayoría de los casos, el método de detección
de \emph{outliers} utilizando el valor $\tau$ resultó efectivo para
identificar los mismos, al menos partiendo de una inspección visual de los
gráficos obtenidos. Sin embargo, hay una excepción notoria es el caso de la
ruta 2A, donde un valor correspondiente a un salto intercontinental fue
omitido por una diferencia mínima.

Observando los gráficos del \texttt{RTT} por saltos normalizado, es posible
advertir una diferencia clara entre los \emph{hops} con \texttt{RTT}s cercanos
a la media de la distribución, que una vez normalizados no exceden en general
un valor de $0.5$, y los \emph{outliers}, que se encuentran siempre por encima
de $1.5$. Así, un valor de corte fijo en $1.5$ podría haber sido utilizado
para identificar a los \emph{outliers} evitando la omisión recién mencionada.
No obstante, al no tener en cuenta la varianza de la distribución, esta
metodología puede fallar de aparecer rutas donde la diferencia no sea tan
marcada, por lo que su efectividad debería ser probada experimentalmente con
mayor profundidad.

En el paper de Jobst \cite{Jobst} se describen unas cuantas ``anomalías'' de
traceroute. En los experimentos realizados aparecieron algunas de ellas,
mientras que otras no. Por supuesto, la denominada \emph{missing hops}
apareció en todos los experimentos, ya que es sumamente común que algunos
\emph{routers} en el camino no respondan. La anomalía denominada \emph{missing
destination} también se presentó (es también muy común que un \emph{host} esté
configurado para no responder los \emph{ICMP Echo}), pero a la hora de elegir
las rutas a analizar, se dejó fuera estos casos.

También se presentaron problemas en la obtención de los RTT (\emph{False
RTTs}). En la ruta C, por ejemplo, se ve un incremento negativo del RTT entre
los hops 6 y 7. En la B2, también, entre los hops 16 y 17. Como se discutió
previamente en el análisis de la ruta A, esto podría ser indicativo de un path
asimétrico entre el envío y la respuesta (toman distintas rutas). No se
hallaron situaciones como las descriptas con respecto a los links MPLS.
