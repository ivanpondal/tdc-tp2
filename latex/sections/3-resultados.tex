\section{Resultados}

\subsection{Ruta A (www.u-tokyo.ac.jp)}

La primera ruta estudiada fue la que lleva al sitio de la
universidad de Tokyo de Japón (\textsc{UTokyo}). La dirección \texttt{URL} de la
misma es \emph{http://www.u-tokyo.ac.jp} que al momento de realizar el
experimento resolvía a la dirección \texttt{IP} \texttt{210.152.135.17}. Este
destino es de particular interés por su lejanía geográfica.

Habiendo ejecutado la herramienta desarrollada, se obtuvo una ruta que llega al
destino con $\texttt{TTL} = 22$.

\begin{figure}[H]
    \figdef[dim]{figures/tokyo_route_table}
    \caption{Ruta obtenida hacia \emph{www.u-tokyo.ac.jp}.}
\end{figure}

De los $22$ paquetes enviados, $4$ de ellos no generaron
respuesta, lo cual representa un $18$\% de los saltos en la ruta. Ignorando
estos puntos para los cuales no fue posible obtener información, la trayectoria
obtenida tiene una longitud de $18$ saltos.

Como se puede observar en la Figura \ref{res:escA:map}, hay un total de 3 saltos
intercontinentales. Para confirmar este hecho se buscó la geolocalización de las
direcciones \texttt{IP} involucradas en diversos sitios que prestan este
servicio y efectivamente corresponden a los países señalados.

\begin{figure*}
    \figdef[dim]{figures/tokyo_route_map}
    \caption{Localización de saltos según geolocalización de direcciones IP para
    el sitio \emph{www.u-tokyo.ac.jp}.}
    \label{res:escA:map}
\end{figure*}

Esto es de particular interés puesto que la comunicación podría haberse
realizado yendo directo a Estados Unidos y luego a Japón, sin embargo, todas las
veces que se observó la trayectoria, la misma contenía el salto a Europa.

\begin{figure}[H]
    \figdef[dim]{figures/tokyo_route_rtt_plot}
    \caption{Valores obtenidos para el \texttt{RTT} entre saltos ($X_i$) de la ruta A.}
    \label{res:escA:rtt}
\end{figure}

Con respecto a los resultados obtenidos mediante el método \emph{Cimbala}, se
pueden observar varias cuestiones. El mismo detecta como \emph{outliers} los
saltos $12$ (\texttt{129.250.2.227}) y $15$ (\texttt{129.250.3.86}). El salto
$12$ efectivamente corresponde al intercontinental de Europa a Estados Unidos.
Por otro lado, el $15$ es perteneciente y le sigue una dirección de Estados Unidos,
con lo cual resulta extraño el hecho de que figure como \emph{outlier}. Sumado a
esto está el hecho de que el resto de los saltos intercontinentales no fueron
detectados.

\begin{figure}[H]
    \figdef[dim]{figures/tokyo_route_norm_rtt_plot}
    \caption{\texttt{RTT} entre saltos normalizado ($\frac{\vert X_i-\bar{X}\vert}{S}$)
    y valor de $\tau_n$ para la ruta A.}
    \label{res:escA:rttnorm}
\end{figure}

Un valor sumamente importante para analizar por qué se obtuvieron tales
resultados es el \texttt{RTT} promedio entre saltos. Para este destino en
particular llaman la atención los siguientes puntos:

\begin{itemize}
    \item Tiempo nulo entre saltos
    \item Tiempo muy bajo entre saltos intercontinentales
\end{itemize}

Para los tiempos nulos entre saltos existen varias explicaciones posibles. Al
medir el tiempo que tomaban en ir y volver los mensajes se observó que en
algunos casos al incrementar el \texttt{TTL}, el paquete tardaba menos. Esto lo
que genera es que al querer tomar la diferencia de tiempo entre estos puntos se
obtenga un resultado negativo. Para evitar esto y poder aplicar el método de
\emph{Cimbala} se tomó la decisión de asignarles un valor nulo. Ahora, que los
paquetes \emph{tarden menos} puede ser consecuencia de diversos factores. Uno es
que el camino por el cual viaja la respuesta del \texttt{ICMP} no tiene por qué
ser el mismo que por el utilizado para llegar al destino.  Esto significa que con un
\texttt{TTL} mayor podría suceder que el mensaje vuelva por un camino menos
congestionado bajando así su \texttt{RTT} al punto donde resulta menor que el
del nodo anterior. Otra posible explicación es la del balanceo de carga, donde
puede ocurrir que para los distintos valores de \texttt{TTL} el paquete enviado
no tenga el mismo recorrido.

El tiempo bajo para saltos intercontinentales también resulta llamativo y podría
explicarse también como el resultado de estar tomando las diferencias de tiempo
sobre valores que no reflejan correctamente las rutas reales. Con $\texttt{TTL}
= 17$ (\texttt{61.200.80.218}) se debería estar midiendo el tiempo del salto
intercontinental de Estados Unidos a Japón, sin embargo en la tabla se puede
observar que la diferencia de tiempo es prácticamente nula. Muy posiblemente lo
que esté ocurriendo en este caso particular es que el valor medido para el
\texttt{TTL} anterior corresponda a una ruta distinta a la que termina siendo
utilizada para el salto intercontinental.


\subsection{Ruta B (www.mpg.de)}

Luego estudiamos dos rutas distintas para llegar desde Argentina a la página web de los Institutos Max Planck (\texttt{www.mpg.de}), en Alemania. Al momento de realizar los experimentos, la IP a la que resuelve ese host es \texttt{134.76.31.198}. Elegimos este destino para hacer una comparación que nos parecía interesante: rutas a través de proveedores comerciales de Internet, y rutas a través de redes académicas como RedClara\footnote{https://www.redclara.net/} y Géant. Para hacer la comparación, ejecutamos la herramienta desde la red de la FCEyN y desde Fibertel, en ambos casos, conectados por Ethernet a cada red.

\subsubsection{Ruta B1 - Internet comercial}


La ruta obtenida cuando la trazamos desde el ISP Fibertel es la siguiente:

\begin{figure}[H]
    \figdef[dim]{figures/maxplanck_desde_fibertel_table}
    \caption{Ruta obtenida hacia \emph{www.mpg.de} desde el ISP Fibertel.}
    \label{res:escB1:table}
\end{figure}

De los 21 saltos que fueron necesarios para llegar a la IP destino, 6 (29\%) no responden los \emph{Time exceeded}. Sin estos hops, la ruta obtenida es de 15 saltos.

\begin{figure}[H]
    \figdef[dim]{figures/maxplanck_desde_fibertel_map}
    \caption{Localización de saltos según geolocalización de direcciones IP para
    el sitio \emph{www.mpg.de} desde el ISP Fibertel.}
    \label{res:escb1:map}
\end{figure}

En la figura \ref{res:escb1:map} se muestra el planisferio con la ubicación detectada con GeoIP para cada salto. Allí se ven dos saltos intercontinentales: de Argentina a EEUU, y de EEUU a Londres, que se corresponden con los hops 10 y 12. Para tener más datos que soporten esta inferencia, podemos mirar los reversos de IP y los promedios de RTT entre saltos. El salto 10, con IP \texttt{67.17.94.249}, tiene como reverso el hostname \texttt{ae1-300G.ar5.MIA1.gblx.net}, lo cual nos hace pensar que es un host en Miami, probablemente de la red de Global Crossing (Level3). El promedio de RTTs del salto 10 es de 158ms (que es un poco alto para lo esperado para un enlace Argentina-Miami, de aprox 130ms\footnote{http://www.verizonenterprise.com/about/network/latency/}). El salto 12 tiene IP \texttt{4.69.154.137} y reverso de IP \texttt{ae-3-80.edge5.Frankfurt1.Level3.net}. GeoIP ubica ese host en Londres, el RTT entre saltos es de unos 108ms, lo cual está entre lo esperado para un enlace EEUU-Europa.

\begin{figure}[H]
    \figdef[dim]{figures/maxplanck_desde_fibertel_rtt_plot}
    \caption{Valores obtenidos para el \texttt{RTT} entre saltos ($X_i$) de la ruta B1.}
    \label{res:escb1:rtt}
\end{figure}

El método de Cimbala detectó el salto intercontinental entre Argentina y EEUU, pero no el salto EEUU-Europa. Si miramos el gráfico \ref{res:escb1:rttnorm}, vemos que el hop 12 quedó justo en el límite por debajo de $\tau_n$. Quizás, al hacer otra medición, una variación mínima podría lograr que se detecte bien el salto.

\begin{figure}[H]
    \figdef[dim]{figures/maxplanck_desde_fibertel_norm_rtt_plot}
    \caption{\texttt{RTT} entre saltos normalizado ($\frac{\vert X_i-\bar{X}\vert}{S}$)
    y valor de $\tau_n$ para la ruta B1.}
    \label{res:escb1:rttnorm}
\end{figure}

\subsubsection{Ruta B2 - Redes Avanzadas}

La ruta obtenida cuando la trazamos desde la red de la UBA fue la siguiente:

\begin{figure}[H]
    \figdef[dim]{figures/maxplanck_desde_uba_table}
    \caption{Ruta obtenida hacia \emph{www.mpg.de} desde la red de la UBA.}
    \label{res:escB2:table}
\end{figure}

Como era esperable, obtuvimos una ruta con menos saltos, ya que se trata de una ruta especial de redes académicas, que cuenta con un enlace transatlántico directo entre Sudamérica y Europa.
De los 19 saltos que fueron necesarios para llegar a la IP destino, 2 (11\%) no responden los \emph{Time exceeded}. Sin estos hops, la ruta obtenida es de 17 saltos.

\begin{figure}[H]
    \figdef[dim]{figures/maxplanck_desde_uba_map}
    \caption{Localización de saltos según geolocalización de direcciones IP para
    el sitio \emph{www.mpg.de} desde la red de la UBA.}
    \label{res:escb2:map}
\end{figure}

En la figura \ref{res:escb2:map} se muestra el planisferio con la ubicación detectada con GeoIP para cada salto. Allí se ve un solo salto intercontinental: de Argentina a Europa, que se corresponde con el hop 9. Para tener más datos que soporten esta inferencia, podemos mirar los reversos de IP y los promedios de RTT entre saltos. El salto 9, con IP \texttt{62.40.124.36}, fue geolocalizado en Londres por la IP, y tiene como reverseo \texttt{redclara.lon.uk.geant.net.}, lo cual es un indicio fuerte de que la detección de GeoIP fue correcta, y coincide con los planos de topología que ofrece RedClara en su sitio web\footnote{https://www.redclara.net/index.php/en/network-and-connectivity/topologia}. El promedio de RTTs del salto es de 182ms, lo cual está dentro de lo esperable para un salto Argentina-Europa.

\begin{figure}[H]
    \figdef[dim]{figures/maxplanck_desde_uba_rtt_plot}
    \caption{Valores obtenidos para el \texttt{RTT} entre saltos ($X_i$) de la ruta B2.}
    \label{res:escb2:rtt}
\end{figure}

El método de Cimbala detectó el salto intercontinental entre Argentina y Londres. Si miramos el gráfico \ref{res:escb2:rtt}, vemos que el único salto que supera los 50ms es el 9, y el siguiente en ms es el salto 8, que por lo que vemos corresponde a un enlace entre Argentina y Brasil (no se ve en el mapa generado por GeoIP, pero el reverso de IP es \texttt{br-ar.redclara.net} y el RTT entre saltos es de ~36ms, cuando un salto a Uruguay suele estar alrededor de los 12ms\footnote{Por ej. el ping a \texttt{rau-ar.redclara.net} da un RTT total de unos 14ms desde la red de UBA}).

\begin{figure}[H]
    \figdef[dim]{figures/maxplanck_desde_uba_norm_rtt_plot}
    \caption{\texttt{RTT} entre saltos normalizado ($\frac{\vert X_i-\bar{X}\vert}{S}$)
    y valor de $\tau_n$ para la ruta B2.}
    \label{res:escb2:rttnorm}
\end{figure}

\subsection{Ruta C (mit.edu)}

Como destino para el tercero de los experimentos de trazado de rutas, se
eligió el Massachusetts Institute of Technology (MIT), ubicado en el este de
los Estados Unidos. El objetivo era observar una ruta previsiblemente corta,
con un único salto intercontinental, y verificar si al reducirse la cantidad
de saltos mejoraba la calidad de los resultados, en particular, disminuyendo
la cantidad de veces en que un salto presentaba un \texttt{RTT}s menor al
salto anterior.

En primera instancia, se intentó trazar la ruta hacia su
sitio web, disponible bajo el dominio \emph{web.mit.edu}. Sin embargo, se
encontró que la ruta observada contaba solo con 7 saltos, y un \emph{ping}
hacia dicho dominio reveló \texttt{RTT}s inferiores a los 20 ms, valores
similares a los observados para otros sitios ubicados en Buenos
Aires\footnote{Por ejemplo, para el sitio de la UBA (\emph{www.uba.ar}) los
valores de \texttt{RTT} obtenidos fueron en general ligeramente mayores.}.
Esto permitió concluir que la ruta no contenía saltos intercontinentales, y
que la \texttt{IP} de destino, pese a estar geolocalizada en los Estados
Unidos, correspondía a un \emph{host} ubicado físicamente en Argentina. Una
posibilidad es que el acceso al sitio sea brindado a través de una CDN
(Content Delivery Network).

Por este motivo, el experimento se realizó con el dominio \emph{mit.edu}, que,
al momento de llevarlo a cabo, resolvía a la dirección \texttt{IP}
\texttt{104.66.69.243}. Se observó la siguiente ruta:

\begin{figure}[H]
    \figdef[dim]{figures/mit_route_table}
    \caption{Ruta obtenida hacia \emph{mit.edu}.}
    \label{res:escC:table}
\end{figure}

Puede verse que se trata de una ruta considerablemente más corta que las
anteriores, con un total de 10 saltos, de los cuales 4 ($40\%$) no
respondieron con un \emph{Time exceeded}. Cabe destacar que, entre los saltos
que sí respondieron, la única anomalía se observa en el \emph{hop} 5, que
presenta un \texttt{RTT} menor al de los \emph{hops} 6 y 7. Los \texttt{RTT}s
de los demás saltos se encuentran en orden creciente. Esto puede contrastarse
con los experimentos anteriores, donde la proporción de saltos donde se
observan inversiones en el orden de los \texttt{RTT}s es considerablemente
mayor.

Los valores medidos para el \texttt{RTT} entre saltos se muestran en
la Figura \ref{res:escC:rtt}. Se evidencia un único valor notoriamente mayor
al resto, que se ubica entre los saltos 7 y 8.

\begin{figure}[H]
    \figdef[dim]{figures/mit_route_rtt_plot}
    \caption{Valores obtenidos para el \texttt{RTT} entre saltos ($X_i$) de la ruta C.}
    \label{res:escC:rtt}
\end{figure}

Esto es reafirmado por el método de detección de \emph{outliers} de Cimbala
(ver Figura \ref{res:escC:rttnorm}), que señala como único \emph{outlier} al
\emph{hop} número 8. Todos los demás valores están por debajo de los 20 ms,
lo cual es indicio de cercanía geográfica entre estos \emph{hops}.

\begin{figure}[H]
    \figdef[dim]{figures/mit_route_norm_rtt_plot}
    \caption{\texttt{RTT} entre saltos normalizado ($\frac{\vert X_i-\bar{X}\vert}{S}$)
    y valor de $\tau_n$ para la ruta C.}
    \label{res:escC:rttnorm}
\end{figure}

En la Figura \ref{res:escC:map} se muestra el resultado de geolocalizar las
\texttt{IP}s de la ruta mediante GeoIP. Claramente, es distinto lo esperado.
En primer lugar, se ubica un salto intercontinental luego de dos \emph{hops}
que contestaron, es decir, entre los \emph{hops} 6 y 7. No obstante, de
un sencillo análisis de los \texttt{RTT}s se deduce que dicho salto debe
estar ubicado entre los \emph{hops} 7 y 8. Realizando una búsqueda
\texttt{DNS} inversa de la \texttt{IP} correspondiente al salto 7, vemos que
corresponde al nombre \texttt{ae6.baires3.bai.seabone.net}, lo cual parece
indicar que está ubicado en Buenos Aires. Esto contradice a GeoIP, que ubica
este \emph{hop} en el centro de Italia, pero se ajusta bien a los resultados
mostrados por el \emph{traceroute}.

\begin{figure}[H]
    \figdef[dim]{figures/mit_route_map}
    \caption{Localización de saltos según geolocalización de direcciones IP para
    el dominio \emph{mit.edu}.}
    \label{res:escC:map}
\end{figure}

En cuanto al \emph{hop} 8, una búsqueda \texttt{DNS} inversa indica que la
autoridad de nombres correspondiente es \texttt{dns-ops.arin.net}
(\emph{American Registry for Internet Numbers}), lo cual parece señalar que
su verdadera ubicación es en los Estados Unidos. Esto coincide con lo indicado
por un servicio alternativo para la geolocalización de \texttt{IP}s,
PlotIP\footnote{http://www.plotip.com/}, que la localiza en Washington, EE.UU.

La dirección \texttt{IP} de destino, \texttt{104.66.69.243}, que GeoIP indica
en Amsterdam, no pudo ser geolocalizada mediante PlotIP. No obstante,
siguiendo la misma línea de razonamiento, y teniendo en cuenta los valores
observados para los \texttt{RTT}s, es razonable concluir que también debe
estar ubicada en los Estados Unidos, lo cual, por otra parte, tiene sentido
dado que corresponde a una universidad estadounidense.

Del análisis anterior podemos destacar cómo la geolocalización de direcciones
\texttt{IP} puede ser sumamente imprecisa y llevar a conclusiones erróneas,
si no se confrontan sus resultados con los producidos mediante técnicas
distintas.
