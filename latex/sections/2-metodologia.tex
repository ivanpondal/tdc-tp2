\section{Metodología}

Para poder elaborar los experimentos presentados en las próximas secciones fue
necesario desarrollar herramientas que nos permitieran estudiar distintos
aspectos de las rutas seleccionadas.

\subsection{Trazado de rutas}

La primer herramienta implementada fue un \texttt{traceroute}. La misma fue
desarrollada utilizando la estrategia de enviar mensajes \texttt{ICMP} del tipo
\emph{Echo Request} incrementando progresivamente el campo \texttt{TTL}
del paquete \texttt{IP}. De esta manera, si el \texttt{TTL} es menor al número
de saltos requeridos para llegar al destino final, existirá un dispositivo en la
ruta para el cual al decrementarlo valdrá cero. Si el nodo tiene habilitado
el protocolo \texttt{ICMP}, responderá con un \emph{Time Exceeded}. Esto permite
conocer para los saltos pertenecientes a la ruta con \texttt{ICMP} activado, sus
direcciones \texttt{IP} y \emph{RTT}.

Dado que la ruta utilizada para llegar a un nodo no es necesariamente
siempre la misma, para cada valor del campo \texttt{TTL} el script desarrollado
recibe como parámetro el número de muestras que se desean tomar. Por defecto se
envían 30 paquetes por \texttt{TTL}. Una vez realizado el muestreo, se toma la
dirección que generó el mayor número de respuestas y sus respectivos \emph{RTT}
como representantes de ese valor particular del campo \texttt{TTL}.

Además fue necesario considerar el caso para el cual el nodo en la ruta no
responde mensajes \texttt{ICMP}. Esto se trató decidiendo que antes de tomar
el número de muestras solicitado, primero se realizaran algunos \emph{Echo
Request} con el único fin de verificar si para el \texttt{TTL} actual se recibe algún
\emph{Time Exceeded} o \emph{Echo Reply}. En caso de no recibir respuesta, se
incrementa en uno el \texttt{TTL}. Tanto el número de intentos como el tiempo a
esperar hasta recibir una respuesta son parámetros de la herramienta que poseen
valores por defecto.

La herramienta concluye cuande se cumple cualquiera de las siguientes
condiciones:

\begin{itemize}
    \item $\texttt{TTL} > 30$
    \item El último salto consultado respondió con \emph{Echo Reply}
\end{itemize}

Toda la creación y recepción de mensajes en la red fue realizado a través de la
biblioteca \textsc{Scapy} para \textsc{Python}.
