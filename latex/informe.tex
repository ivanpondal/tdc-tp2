\documentclass[%
    final,
    notitlepage,
    narroweqnarray,
    inline,
    twoside,
]{ieee}
\usepackage{ieeefig}
\usepackage{pgfplots}
\pgfplotsset{compat=newest}
\usepackage[utf8]{inputenc}
\usepackage[spanish]{babel}
\usepackage{amsmath}
\usepackage{float}
\usepackage{siunitx} % http://mirrors.ctan.org/install/macros/latex/contrib/siunitx.tds.zip
\sisetup{
    binary-units    = true,
    round-mode      = places,
    round-precision = 2,
    output-decimal-marker = {,}
}

% Style to select only points from #1 to #2 (inclusive)
\pgfplotsset{select coords between index/.style 2 args={
    x filter/.code={
        \ifnum\coordindex<#1\def\pgfmathresult{}\fi
        \ifnum\coordindex>#2\def\pgfmathresult{}\fi
    }
}}

\def\keywordsname{Palabras clave}

\usepackage[hidelinks]{hyperref}

\begin{document}

\title[Trabajo Práctico 2: Rutas en Internet]{%
       Trabajo Práctico 2: \emph{Rutas en Internet}}

\author[FRIZZO, MONTEPAGANO, PONDAL]{Franco Frizzo, Pablo Montepagano
\and{}e Iván Pondal}

\journal{Teoría de las Comunicaciones (DC - FCEyN - UBA)}
\titletext{, Trabajo Práctico 2: \emph{Rutas en Internet}}

\maketitle

\begin{abstract}
[Inserte resumen aquí]
\end{abstract}

\begin{keywords}
[palabra clave 1, palabra clave 2, palabra clave 3]
\end{keywords}

\section{Introducción}

Frente al rápido crecimiento de \emph{Internet} y su enorme infraestructura
subyacente, las rutas utilizadas para el tráfico de paquetes entre nodos, con su
gran número de factores que influyen en su naturaleza cambiante, resultan ser un
objeto de gran interés.

El estudio realizado se concentra en el uso del protocolo de control
\texttt{ICMP} para obtener los nodos involucrados en una ruta y la estimación de
saltos intercontinentales mediante la medición de los \texttt{RTT} de los
paquetes utilizados. En el mismo, se experimentó utilizando como destinos los
sitios web de tres universidades del mundo para así tener la posibilidad de
obtener varios saltos intercontinentales y rutas de distintas longitudes.

Para cada ruta analizada se presenta un mapa con la geolocalización de las
direcciones \texttt{IP} pertenecientes a la traza junto a un desarrollo sobre el
nivel de efectividad de los métodos aplicados tanto para estimar la estructura
de los caminos como la detección de saltos intercontinentales.

\section{Metodología}

Para poder elaborar los experimentos presentados en las próximas secciones fue
necesario desarrollar herramientas que nos permitieran estudiar distintos
aspectos de las rutas seleccionadas.

\subsection{Trazado de rutas}

La primer herramienta implementada fue un \texttt{traceroute}. La misma fue
desarrollada utilizando la estrategia de enviar mensajes \texttt{ICMP} del tipo
\emph{Echo Request} incrementando progresivamente el campo \texttt{TTL}
del paquete \texttt{IP}. De esta manera, si el \texttt{TTL} es menor al número
de saltos requeridos para llegar al destino final, existirá un dispositivo en la
ruta para el cual al decrementarlo valdrá cero. Si el nodo tiene habilitado
el protocolo \texttt{ICMP}, responderá con un \emph{Time Exceeded}. Esto permite
conocer para los saltos pertenecientes a la ruta con \texttt{ICMP} activado, sus
direcciones \texttt{IP} y \emph{RTT}.

Dado que la ruta utilizada para llegar a un nodo no es necesariamente
siempre la misma, para cada valor del campo \texttt{TTL} el script desarrollado
recibe como parámetro el número de muestras que se desean tomar. Por defecto se
envían 30 paquetes por \texttt{TTL}. Una vez realizado el muestreo, se toma la
dirección que generó el mayor número de respuestas y sus respectivos \emph{RTT}
como representantes de ese valor particular del campo \texttt{TTL}.

Además fue necesario considerar el caso para el cual el nodo en la ruta no
responde mensajes \texttt{ICMP}. Esto se trató decidiendo que antes de tomar
el número de muestras solicitado, primero se realizaran algunos \emph{Echo
Request} con el único fin de verificar si para el \texttt{TTL} actual se recibe algún
\emph{Time Exceeded} o \emph{Echo Reply}. En caso de no recibir respuesta, se
incrementa en uno el \texttt{TTL}. Tanto el número de intentos como el tiempo a
esperar hasta recibir una respuesta son parámetros de la herramienta que poseen
valores por defecto.

La herramienta concluye cuande se cumple cualquiera de las siguientes
condiciones:

\begin{itemize}
    \item $\texttt{TTL} > 30$
    \item El último salto consultado respondió con \emph{Echo Reply}
\end{itemize}

Toda la creación y recepción de mensajes en la red fue realizado a través de la
biblioteca \textsc{Scapy} para \textsc{Python}.

\section{Resultados}

\subsection{Ruta 1 (www.u-tokyo.ac.jp)}

La primera ruta estudiada fue la que lleva al sitio de la
universidad de Tokyo de Japón (\textsc{UTokyo}). La dirección \texttt{URL} de la
misma es \emph{http://www.u-tokyo.ac.jp} que al momento de realizar el
experimento resolvía a la dirección \texttt{IP} \texttt{210.152.135.17}. Este
destino es de particular interés por su lejanía geográfica.

Habiendo ejecutado la herramienta desarrollada, se obtuvo una ruta que llega al
destino con $\texttt{TTL} = 22$.

\begin{figure}[H]
    \figdef[dim]{figures/tokyo_route_table}
    \caption{Dirección \texttt{IP} para cada \texttt{TTL}.}
\end{figure}

De los $22$ paquetes enviados, $4$ de ellos no generaron
respuesta, lo cual representa un $18$\% de los saltos en la ruta. Ignorando
estos puntos para los cuales no fue posible obtener información, la trayectoria
obtenida tiene una longitud de $18$ saltos.

Como se puede observar en la Figura \ref{res:esc1:map}, hay un total de 3 saltos
intercontinentales. Para confirmar este hecho se buscó la geolocalización de las
direcciones \texttt{IP} involucradas en diversos sitios que prestan este
servicio y efectivamente corresponden a los países señalados.

\begin{figure*}
    \figdef[dim]{figures/tokyo_route_map}
    \caption{Localización de saltos según geolocalización de direcciones IP para
    el sitio \emph{www.u-tokyo.ac.jp}.}
    \label{res:esc1:map}
\end{figure*}

Esto es de particular interés puesto que la comunicación podría haberse
realizado yendo directo a Estados Unidos y luego a Japón, sin embargo, todas las
veces que se observó la trayectoria, la misma contenía el salto a Europa.

\begin{figure}[H]
    \figdef[dim]{figures/tokyo_route_rtt_plot}
    \caption{Valores obtenidos para el \texttt{RTT} entre saltos ($X_i$) de la ruta 1.}
    \label{res:esc1:rtt}
\end{figure}

Con respecto a los resultados obtenidos mediante el método \emph{Cimbala}, se
pueden observar varias cuestiones. El mismo detecta como \emph{outliers} los
saltos $11$ (\texttt{129.250.2.227}) y $14$ (\texttt{129.250.3.86}). El salto
$11$ efectivamente corresponde al intercontinental de Europa a Estados Unidos.
Por otro lado, el $14$ es perteneciente y le sigue una dirección de Estados Unidos,
con lo cual resulta extraño el hecho de que figure como \emph{outlier}. Sumado a
esto está el hecho de que el resto de los saltos intercontinentales no fueron
detectados.

\begin{figure}[H]
    \figdef[dim]{figures/tokyo_route_norm_rtt_plot}
    \caption{\texttt{RTT} entre saltos normalizado ($\frac{X_i-\bar{X}}{S}$)
    y valor de $\tau_n$ para la ruta 1.}
    \label{res:esc1:rtt}
\end{figure}

Un valor sumamente importante para analizar por qué se obtuvieron tales
resultados es el \texttt{RTT} promedio entre saltos. Para este destino en
particular llaman la atención los siguientes puntos:

\begin{itemize}
    \item Tiempo nulo entre saltos
    \item Tiempo muy bajo entre saltos intercontinentales
\end{itemize}

Para los tiempos nulos entre saltos existen varias explicaciones posibles. Al
medir el tiempo que tomaban en ir y volver los mensajes se observó que en
algunos casos al incrementar el \texttt{TTL}, el paquete tardaba menos. Esto lo
que genera es que al querer tomar la diferencia de tiempo entre estos puntos se
obtenga un resultado negativo. Para evitar esto y poder aplicar el método de
\emph{Cimbala} se tomó la decisión de asignarles un valor nulo. Ahora, que los
paquetes \emph{tarden menos} puede ser consecuencia de diversos factores. Uno es
que el camino por el cual viaja la respuesta del \texttt{ICMP} no tiene por qué
ser el mismo que por el utilizado para llegar al destino.  Esto significa que con un
\texttt{TTL} mayor podría suceder que el mensaje vuelva por un camino menos
congestionado bajando así su \texttt{RTT} al punto donde resulta menor que el
del nodo anterior. Otra posible explicación es la del balanceo de carga, donde
puede ocurrir que para los distintos valores de \texttt{TTL} el paquete enviado
no tenga el mismo recorrido.

El tiempo bajo para saltos intercontinentales también resulta llamativo y podría
explicarse también como el resultado de estar tomando las diferencias de tiempo
sobre valores que no reflejan correctamente las rutas reales. Con $\texttt{TTL}
= 17$ (\texttt{61.200.80.218}) se debería estar midiendo el tiempo del salto
intercontinental de Estados Unidos a Japón, sin embargo en la tabla se puede
observar que la diferencia de tiempo es prácticamente nula. Muy posiblemente lo
que esté ocurriendo en este caso particular es que el valor medido para el
\texttt{TTL} anterior corresponda a una ruta distinta a la que termina siendo
utilizada para el salto intercontinental.


\subsection{Ruta 2 (www.mpg.de)}

Luego estudiamos dos rutas distintas para llegar desde Argentina a la página web de los Institutos Max Planck (\texttt{www.mpg.de}), en Alemania. Al momento de realizar los experimentos, la IP a la que resuelve ese host es \texttt{134.76.31.198}. Elegimos este destino para hacer una comparación que nos parecía interesante: rutas a través de proveedores comerciales de Internet, y rutas a través de redes académicas como RedClara y Géant. Para hacer la comparación, ejecutamos la herramienta desde la red de la FCEyN y desde Fibertel, en ambos casos, conectados por Ethernet a cada red.

\subsubsection{Ruta 2A - Internet comercial}


La ruta obtenida cuando la trazamos desde el ISP Fibertel es la siguiente:

\begin{Verbatim}[fontsize=\scriptsize]
Hop Most frequent IP  Avg          Delta         Std     Loss%
 1: 192.168.0.1       75.269 ms                  6.878  0.00% 
 2: N/A            
 3: N/A            
 4: N/A            
 5: N/A            
 6: 200.89.161.77     86.499 ms  (d  11.231 ms)  9.165  0.00% 
 7: 200.89.165.197    85.378 ms  (d   0.000 ms)  8.914  0.00% 
 8: 200.89.165.222    86.714 ms  (d   1.335 ms)  9.365  0.00% 
 9: N/A            
10: 67.17.94.249     245.380 ms  (d 158.666 ms) 14.962  0.00% <<< outlier
11: N/A            
12: 4.69.154.137     354.046 ms  (d 108.666 ms) 14.393  0.00% 
13: 4.69.154.137     359.716 ms  (d   5.670 ms) 13.302  0.00% 
14: 212.162.4.6      347.716 ms  (d   0.000 ms) 22.849  0.00% 
15: 188.1.144.134    332.160 ms  (d   0.000 ms) 16.108  0.00% 
16: 188.1.231.126    332.714 ms  (d   0.554 ms) 17.657  0.00% 
17: 134.76.250.251   331.388 ms  (d   0.000 ms) 13.188 20.00% 
18: 134.76.250.4     351.826 ms  (d  20.437 ms) 13.845  0.00% 
19: 134.76.250.238   361.931 ms  (d  10.105 ms) 14.248  0.00% 
20: 134.76.249.101   349.935 ms  (d   0.000 ms) 14.059  0.00% 
21: 134.76.31.198    326.050 ms  (d   0.000 ms) 11.225  0.00%
\end{Verbatim}

De los 21 saltos que fueron necesarios para llegar a la IP destino, 6 (29\%) no responden los \emph{Time exceeded}. Sin estos hops, la ruta obtenida es de 15 saltos.

\begin{figure*}
    \figdef[dim]{figures/maxplanck_desde_fibertel_map}
    \caption{Localización de saltos según geolocalización de direcciones IP para
    el sitio \emph{www.mpg.de} desde el ISP Fibertel.}
    \label{res:esc2a:map}
\end{figure*}

En la figura \ref{res:esc2a:map} se muestra el planisferio con la ubicación detectada con GeoIP para cada salto. Allí se ven dos saltos intercontinentales: de Argentina a EEUU, y de EEUU a Londres, que se corresponden con los hops 10 y 12. Para tener más datos que soporten esta inferencia, podemos mirar los reversos de IP y los promedios de RTT entre saltos. El salto 10, con IP \texttt{67.17.94.249}, tiene como reverso el hostname \texttt{ae1-300G.ar5.MIA1.gblx.net}, lo cual nos hace pensar que es un host de una punta de un enlace de 300Gbps entre Argentina y Miami.



\bibliographystyle{IEEEbib}
\bibliography{informe}

\end{document}
