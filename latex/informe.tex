\documentclass[%
    final,
    notitlepage,
    narroweqnarray,
    inline,
    twoside,
]{ieee}
\usepackage{ieeefig}
\usepackage{pgfplots}
\pgfplotsset{compat=newest}
\usepackage[utf8]{inputenc}
\usepackage[spanish]{babel}
\usepackage{amsmath}
\usepackage{float}
\usepackage{fancyvrb}
\usepackage{siunitx} % http://mirrors.ctan.org/install/macros/latex/contrib/siunitx.tds.zip
\sisetup{
    binary-units    = true,
    round-mode      = places,
    round-precision = 2,
    output-decimal-marker = {,}
}

% Style to select only points from #1 to #2 (inclusive)
\pgfplotsset{select coords between index/.style 2 args={
    x filter/.code={
        \ifnum\coordindex<#1\def\pgfmathresult{}\fi
        \ifnum\coordindex>#2\def\pgfmathresult{}\fi
    }
}}

\def\keywordsname{Palabras clave}

\usepackage[hidelinks]{hyperref}

\newcommand\ThompsonTau{0}

\begin{document}

\nocite{Jobst}

\title[Trabajo Práctico 2: Rutas en Internet]{%
       Trabajo Práctico 2: \emph{Rutas en Internet}}

\author[FRIZZO, MONTEPAGANO, PONDAL]{Franco Frizzo, Pablo Montepagano
\and{}e Iván Pondal}

\journal{Teoría de las Comunicaciones (DC - FCEyN - UBA)}
\titletext{, Trabajo Práctico 2: \emph{Rutas en Internet}}

\maketitle

\begin{abstract}
    En este trabajo se estudió el trazado de rutas en \emph{Internet} mediante el uso
    del protocolo \texttt{ICMP}. Se presenta la implementación de un método de
    detección de saltos intercontinentales en base al \emph{RTT} de los paquetes y
    un análisis al mismo aplicado a las rutas obtenidas para los sitios web de
    tres universidades del mundo.
\end{abstract}

\begin{keywords}
internet, rutas, saltos intercontinentales, ICMP, traceroute.
\end{keywords}

\section{Introducción}
[Texto de la introducción]

\section{Metodología}

Para poder elaborar los experimentos presentados en las próximas secciones fue
necesario desarrollar herramientas que nos permitieran estudiar distintos
aspectos de las rutas seleccionadas.

\subsection{Trazado de rutas}

La primera herramienta implementada fue un \texttt{traceroute}. La misma fue
desarrollada utilizando la estrategia de enviar mensajes \texttt{ICMP} del tipo
\emph{Echo Request} incrementando progresivamente el campo \texttt{TTL}
del paquete \texttt{IP}. De esta manera, si el \texttt{TTL} es menor al número
de saltos requeridos para llegar al destino final, existirá un dispositivo en la
ruta para el cual al decrementarlo valdrá cero. Si el nodo tiene habilitado
el protocolo \texttt{ICMP}, responderá con un \emph{Time Exceeded}. Esto permite
conocer para los saltos pertenecientes a la ruta con \texttt{ICMP} activado, sus
direcciones \texttt{IP} y \emph{RTT}.

Dado que la ruta utilizada para llegar a un nodo no es necesariamente
siempre la misma, para cada valor del campo \texttt{TTL} el script desarrollado
recibe como parámetro el número de muestras que se desean tomar. Por defecto se
envían 30 paquetes por \texttt{TTL}. Una vez realizado el muestreo, se toma la
dirección que generó el mayor número de respuestas y sus respectivos \emph{RTT}
como representantes de ese valor particular del campo \texttt{TTL}.

Además fue necesario considerar el caso para el cual el nodo en la ruta no
responde mensajes \texttt{ICMP}. Esto se trató decidiendo que antes de tomar
el número de muestras solicitado, primero se realizaran algunos \emph{Echo
Request} con el único fin de verificar si para el \texttt{TTL} actual se recibe algún
\emph{Time Exceeded} o \emph{Echo Reply}. En caso de no recibir respuesta, se
incrementa en uno el \texttt{TTL}. Tanto el número de intentos como el tiempo a
esperar hasta recibir una respuesta son parámetros de la herramienta que poseen
valores por defecto.

La herramienta concluye cuande se cumple cualquiera de las siguientes
condiciones:

\begin{itemize}
    \item $\texttt{TTL} > 30$
    \item El último salto consultado respondió con \emph{Echo Reply}
\end{itemize}

Toda la creación y recepción de mensajes en la red fue realizado a través de la
biblioteca \textsc{Scapy} para \textsc{Python}.

\subsection{Identificación de saltos intercontinentales}

A partir de los resultados obtenidos por la herramienta anterior, en
particular, los valores del \emph{RTT} para cada uno de los saltos, se
implementó una segunda herramienta en \textsc{Python} con el objetivo de
identificar de forma automatizada cuáles de los mismos corresponden a saltos
intercontinentales. Para hacerlo, se partió de la hipótesis de que, dada la
considerable longitud de dichos saltos, el valor su \emph{RTT} sería
considerablemente mayor. Así, considerando al conjunto de los valores del
\emph{RTT} de una ruta como una muestra aleatoria, podrían atribuirse a saltos
intercontinentales los \emph{outliers} de dicha muestra.

Para la identificación de los \emph{outliers}, se partió de lo propuesto por
Cimbala en \cite{Cimbala}. Allí se expone la \emph{técnica modificada de
Thompson}, que consiste en calcular un valor crítico para la muestra
(basado en su desvío estándar) y considerar como \emph{outliers} a
las mediciones cuya distancia a la media sea mayor que este valor. Dicho
valor se calcula a partir de la siguiente fórmula:
\[ \tau_n = \frac{t_{\alpha / 2} \cdot (n - 1)}
    {\sqrt{n} \cdot \sqrt{n - 2 + \left(t_{\alpha / 2}\right)^2}} \]
donde $n$ es la cantidad de mediciones y $t_{\alpha / 2}$ es el valor crítico de
la distribución t de Student, con $n - 2$ grados de libertad, para $\alpha =
0.5$. Se considera que un valor $X_i$ es \emph{outlier} cuando
\[ \vert X_i - \bar{X} \vert > S \cdot \tau_n \]

Cimbala propone la aplicación del método en forma iterativa: se evalúa si debe
o no eliminarse el valor más extremo de la muestra, y en caso afirmativo, se
vuelven a calcular $\tau_n$, $\bar{X}$ y $S$ para la muestra modificada. En
este trabajo, también se evaluó la aplicación no iterativa del método. Esto
reduce la sensibilidad de la técnica, ya que evita ir reduciendo el desvío
estándar de la muestra y, por lo tanto, el margen de valores que se consideran
aceptables.

\section{Resultados}

\subsection{Ruta A (www.u-tokyo.ac.jp)}

La primera ruta estudiada fue la que lleva al sitio de la
universidad de Tokyo de Japón (\textsc{UTokyo}). La dirección \texttt{URL} de la
misma es \emph{http://www.u-tokyo.ac.jp} que al momento de realizar el
experimento resolvía a la dirección \texttt{IP} \texttt{210.152.135.17}. Este
destino es de particular interés por su lejanía geográfica.

Habiendo ejecutado la herramienta desarrollada, se obtuvo una ruta que llega al
destino con $\texttt{TTL} = 22$.

\begin{figure}[H]
    \figdef[dim]{figures/tokyo_route_table}
    \caption{Dirección \texttt{IP} para cada \texttt{TTL}.}
\end{figure}

De los $22$ paquetes enviados, $4$ de ellos no generaron
respuesta, lo cual representa un $18$\% de los saltos en la ruta. Ignorando
estos puntos para los cuales no fue posible obtener información, la trayectoria
obtenida tiene una longitud de $18$ saltos.

Como se puede observar en la Figura \ref{res:escA:map}, hay un total de 3 saltos
intercontinentales. Para confirmar este hecho se buscó la geolocalización de las
direcciones \texttt{IP} involucradas en diversos sitios que prestan este
servicio y efectivamente corresponden a los países señalados.

\begin{figure*}
    \figdef[dim]{figures/tokyo_route_map}
    \caption{Localización de saltos según geolocalización de direcciones IP para
    el sitio \emph{www.u-tokyo.ac.jp}.}
    \label{res:escA:map}
\end{figure*}

Esto es de particular interés puesto que la comunicación podría haberse
realizado yendo directo a Estados Unidos y luego a Japón, sin embargo, todas las
veces que se observó la trayectoria, la misma contenía el salto a Europa.

\begin{figure}[H]
    \figdef[dim]{figures/tokyo_route_rtt_plot}
    \caption{Valores obtenidos para el \texttt{RTT} entre saltos ($X_i$) de la ruta A.}
    \label{res:escA:rtt}
\end{figure}

Con respecto a los resultados obtenidos mediante el método \emph{Cimbala}, se
pueden observar varias cuestiones. El mismo detecta como \emph{outliers} los
saltos $11$ (\texttt{129.250.2.227}) y $14$ (\texttt{129.250.3.86}). El salto
$11$ efectivamente corresponde al intercontinental de Europa a Estados Unidos.
Por otro lado, el $14$ es perteneciente y le sigue una dirección de Estados Unidos,
con lo cual resulta extraño el hecho de que figure como \emph{outlier}. Sumado a
esto está el hecho de que el resto de los saltos intercontinentales no fueron
detectados.

\begin{figure}[H]
    \figdef[dim]{figures/tokyo_route_norm_rtt_plot}
    \caption{\texttt{RTT} entre saltos normalizado ($\frac{X_i-\bar{X}}{S}$)
    y valor de $\tau_n$ para la ruta A.}
    \label{res:escA:rttnorm}
\end{figure}

Un valor sumamente importante para analizar por qué se obtuvieron tales
resultados es el \texttt{RTT} promedio entre saltos. Para este destino en
particular llaman la atención los siguientes puntos:

\begin{itemize}
    \item Tiempo nulo entre saltos
    \item Tiempo muy bajo entre saltos intercontinentales
\end{itemize}

Para los tiempos nulos entre saltos existen varias explicaciones posibles. Al
medir el tiempo que tomaban en ir y volver los mensajes se observó que en
algunos casos al incrementar el \texttt{TTL}, el paquete tardaba menos. Esto lo
que genera es que al querer tomar la diferencia de tiempo entre estos puntos se
obtenga un resultado negativo. Para evitar esto y poder aplicar el método de
\emph{Cimbala} se tomó la decisión de asignarles un valor nulo. Ahora, que los
paquetes \emph{tarden menos} puede ser consecuencia de diversos factores. Uno es
que el camino por el cual viaja la respuesta del \texttt{ICMP} no tiene por qué
ser el mismo que por el utilizado para llegar al destino.  Esto significa que con un
\texttt{TTL} mayor podría suceder que el mensaje vuelva por un camino menos
congestionado bajando así su \texttt{RTT} al punto donde resulta menor que el
del nodo anterior. Otra posible explicación es la del balanceo de carga, donde
puede ocurrir que para los distintos valores de \texttt{TTL} el paquete enviado
no tenga el mismo recorrido.

El tiempo bajo para saltos intercontinentales también resulta llamativo y podría
explicarse también como el resultado de estar tomando las diferencias de tiempo
sobre valores que no reflejan correctamente las rutas reales. Con $\texttt{TTL}
= 17$ (\texttt{61.200.80.218}) se debería estar midiendo el tiempo del salto
intercontinental de Estados Unidos a Japón, sin embargo en la tabla se puede
observar que la diferencia de tiempo es prácticamente nula. Muy posiblemente lo
que esté ocurriendo en este caso particular es que el valor medido para el
\texttt{TTL} anterior corresponda a una ruta distinta a la que termina siendo
utilizada para el salto intercontinental.


\subsection{Ruta B (www.mpg.de)}

Luego estudiamos dos rutas distintas para llegar desde Argentina a la página web de los Institutos Max Planck (\texttt{www.mpg.de}), en Alemania. Al momento de realizar los experimentos, la IP a la que resuelve ese host es \texttt{134.76.31.198}. Elegimos este destino para hacer una comparación que nos parecía interesante: rutas a través de proveedores comerciales de Internet, y rutas a través de redes académicas como RedClara\footnote{https://www.redclara.net/} y Géant. Para hacer la comparación, ejecutamos la herramienta desde la red de la FCEyN y desde Fibertel, en ambos casos, conectados por Ethernet a cada red.

\subsubsection{Ruta B1 - Internet comercial}


La ruta obtenida cuando la trazamos desde el ISP Fibertel es la siguiente:

\begin{Verbatim}[fontsize=\scriptsize]
Hop Most frequent IP  Avg          Delta         Std     Loss%
 1: 192.168.0.1       75.269 ms                  6.878  0.00% 
 2: N/A            
 3: N/A            
 4: N/A            
 5: N/A            
 6: 200.89.161.77     86.499 ms  (d  11.231 ms)  9.165  0.00% 
 7: 200.89.165.197    85.378 ms  (d   0.000 ms)  8.914  0.00% 
 8: 200.89.165.222    86.714 ms  (d   1.335 ms)  9.365  0.00% 
 9: N/A            
10: 67.17.94.249     245.380 ms  (d 158.666 ms) 14.962  0.00% <<< outlier
11: N/A            
12: 4.69.154.137     354.046 ms  (d 108.666 ms) 14.393  0.00% 
13: 4.69.154.137     359.716 ms  (d   5.670 ms) 13.302  0.00% 
14: 212.162.4.6      347.716 ms  (d   0.000 ms) 22.849  0.00% 
15: 188.1.144.134    332.160 ms  (d   0.000 ms) 16.108  0.00% 
16: 188.1.231.126    332.714 ms  (d   0.554 ms) 17.657  0.00% 
17: 134.76.250.251   331.388 ms  (d   0.000 ms) 13.188 20.00% 
18: 134.76.250.4     351.826 ms  (d  20.437 ms) 13.845  0.00% 
19: 134.76.250.238   361.931 ms  (d  10.105 ms) 14.248  0.00% 
20: 134.76.249.101   349.935 ms  (d   0.000 ms) 14.059  0.00% 
21: 134.76.31.198    326.050 ms  (d   0.000 ms) 11.225  0.00%
\end{Verbatim}

De los 21 saltos que fueron necesarios para llegar a la IP destino, 6 (29\%) no responden los \emph{Time exceeded}. Sin estos hops, la ruta obtenida es de 15 saltos.

\begin{figure}[H]
    \figdef[dim]{figures/maxplanck_desde_fibertel_map}
    \caption{Localización de saltos según geolocalización de direcciones IP para
    el sitio \emph{www.mpg.de} desde el ISP Fibertel.}
    \label{res:escb1:map}
\end{figure}

En la figura \ref{res:escb1:map} se muestra el planisferio con la ubicación detectada con GeoIP para cada salto. Allí se ven dos saltos intercontinentales: de Argentina a EEUU, y de EEUU a Londres, que se corresponden con los hops 10 y 12. Para tener más datos que soporten esta inferencia, podemos mirar los reversos de IP y los promedios de RTT entre saltos. El salto 10, con IP \texttt{67.17.94.249}, tiene como reverso el hostname \texttt{ae1-300G.ar5.MIA1.gblx.net}, lo cual nos hace pensar que es un host en Miami, probablemente de la red de Global Crossing (Level3). El promedio de RTTs del salto 10 es de 158ms (que es un poco alto para lo esperado para un enlace Argentina-Miami, de aprox 130ms\footnote{http://www.verizonenterprise.com/about/network/latency/}). El salto 12 tiene IP \texttt{4.69.154.137} y reverso de IP \texttt{ae-3-80.edge5.Frankfurt1.Level3.net}. GeoIP ubica ese host en Londres, el RTT entre saltos es de unos 108ms, lo cual está entre lo esperado para un enlace EEUU-Europa.

\begin{figure}[H]
    \figdef[dim]{figures/maxplanck_desde_fibertel_rtt_plot}
    \caption{Valores obtenidos para el \texttt{RTT} entre saltos ($X_i$) de la ruta 2A.}
    \label{res:escb1:rtt}
\end{figure}

El método de Cimbala detectó el salto intercontinental entre Argentina y EEUU, pero no el salto EEUU-Europa. Si miramos el gráfico \ref{res:escb1:rttnorm}, vemos que el hop 12 quedó justo en el límite por debajo de $\tau_n$. Quizás, al hacer otra medición, una variación mínima podría lograr que se detecte bien el salto.

\begin{figure}[H]
    \figdef[dim]{figures/maxplanck_desde_fibertel_norm_rtt_plot}
    \caption{\texttt{RTT} entre saltos normalizado ($\frac{X_i-\bar{X}}{S}$)
    y valor de $\tau_n$ para la ruta 2A.}
    \label{res:escb1:rttnorm}
\end{figure}

\subsubsection{Ruta B2 - Redes Avanzadas}

La ruta obtenida cuando la trazamos desde la red de la UBA fue la siguiente:

\begin{Verbatim}[fontsize=\scriptsize]
Hop Most frequent IP  Avg          Delta         Std     Loss%
 1: 157.92.32.99      66.945 ms                  7.966  0.00% 
 2: N/A            
 3: 157.92.47.98      72.727 ms  (d   5.782 ms)  9.598  0.00% 
 4: 157.92.47.2       68.480 ms  (d   0.000 ms)  7.018  0.00% 
 5: N/A            
 6: 168.96.0.18       66.386 ms  (d   0.000 ms)  6.313  0.00% 
 7: 200.0.204.154     73.264 ms  (d   6.877 ms) 10.241  0.00% 
 8: 200.0.204.63     109.706 ms  (d  36.442 ms)  8.238  0.00% 
 9: 62.40.124.36     292.699 ms  (d 182.993 ms)  7.936  0.00% <<< outlier
10: 62.40.98.81      301.031 ms  (d   8.332 ms)  7.720  0.00% 
11: 62.40.98.61      308.358 ms  (d   7.327 ms)  7.186  0.00% 
12: 62.40.112.146    319.237 ms  (d  10.879 ms)  6.780  0.00% 
13: 188.1.144.137    321.472 ms  (d   2.235 ms) 10.064  0.00% 
14: 188.1.231.126    319.931 ms  (d   0.000 ms) 10.518  0.00% 
15: 134.76.250.251   323.690 ms  (d   3.759 ms)  8.331 10.00% 
16: 134.76.250.4     325.366 ms  (d   1.675 ms) 12.971  0.00% 
17: 134.76.147.17    320.704 ms  (d   0.000 ms)  7.337  0.00% 
18: 134.76.249.201   321.922 ms  (d   1.218 ms)  8.492  0.00% 
19: 134.76.31.198    320.150 ms  (d   0.000 ms)  7.008  0.00%
\end{Verbatim}

Como era esperable, obtuvimos una ruta con menos saltos, ya que se trata de una ruta especial de redes académicas, que cuenta con un enlace transatlántico directo entre Sudamérica y Europa.
De los 19 saltos que fueron necesarios para llegar a la IP destino, 2 (11\%) no responden los \emph{Time exceeded}. Sin estos hops, la ruta obtenida es de 17 saltos.

\begin{figure}[H]
    \figdef[dim]{figures/maxplanck_desde_uba_map}
    \caption{Localización de saltos según geolocalización de direcciones IP para
    el sitio \emph{www.mpg.de} desde la red de la UBA.}
    \label{res:escb2:map}
\end{figure}

En la figura \ref{res:escb2:map} se muestra el planisferio con la ubicación detectada con GeoIP para cada salto. Allí se ve un solo salto intercontinental: de Argentina a Europa, que se corresponde con el hop 9. Para tener más datos que soporten esta inferencia, podemos mirar los reversos de IP y los promedios de RTT entre saltos. El salto 9, con IP \texttt{62.40.124.36}, fue geolocalizado en Londres por la IP, y tiene como reverseo \texttt{redclara.lon.uk.geant.net.}, lo cual es un indicio fuerte de que la detección de GeoIP fue correcta, y coincide con los planos de topología que ofrece RedClara en su sitio web\footnote{https://www.redclara.net/index.php/en/network-and-connectivity/topologia}. El promedio de RTTs del salto es de 182ms, lo cual está dentro de lo esperable para un salto Argentina-Europa.

\begin{figure}[H]
    \figdef[dim]{figures/maxplanck_desde_uba_rtt_plot}
    \caption{Valores obtenidos para el \texttt{RTT} entre saltos ($X_i$) de la ruta 2B.}
    \label{res:escb2:rtt}
\end{figure}

El método de Cimbala detectó el salto intercontinental entre Argentina y Londres. Si miramos el gráfico \ref{res:escb2:rtt}, vemos que el único salto que supera los 50ms es el 9, y el siguiente en ms es el salto 8, que por lo que vemos corresponde a un enlace entre Argentina y Brasil (no se ve en el mapa generado por GeoIP, pero el reverso de IP es \texttt{br-ar.redclara.net} y el RTT entre saltos es de ~36ms, cuando un salto a Uruguay suele estar alrededor de los 12ms\footnote{Por ej. el ping a \texttt{rau-ar.redclara.net} da un RTT total de unos 14ms desde la red de UBA}).

\begin{figure}[H]
    \figdef[dim]{figures/maxplanck_desde_uba_norm_rtt_plot}
    \caption{\texttt{RTT} entre saltos normalizado ($\frac{X_i-\bar{X}}{S}$)
    y valor de $\tau_n$ para la ruta 2B.}
    \label{res:escb2:rttnorm}
\end{figure}

\subsection{Ruta C (mit.edu)}

Como destino para el tercero de los experimentos de trazado de rutas, se
eligió el Massachusetts Institute of Technology (MIT), ubicado en el este de
los Estados Unidos. El objetivo era observar una ruta previsiblemente corta,
con un único salto intercontinental, y verificar si al reducirse la cantidad
de saltos mejoraba la calidad de los resultados, en particular, disminuyendo
la cantidad de veces en que un salto presentaba un \texttt{RTT}s menor al
salto anterior.

En primera instancia, se intentó trazar la ruta hacia su
sitio web, disponible bajo el dominio \emph{web.mit.edu}. Sin embargo, se
encontró que la ruta observada contaba solo con 7 saltos, y un \emph{ping}
hacia dicho dominio reveló \texttt{RTT}s inferiores a los 20 ms, valores
similares a los observados para otros sitios ubicados en Buenos
Aires\footnote{Por ejemplo, para el sitio de la UBA (\emph{www.uba.ar}) los
valores de \texttt{RTT} obtenidos fueron en general ligeramente mayores.}.
Esto permitió concluir que la ruta no contenía saltos intercontinentales, y
que la \texttt{IP} de destino, pese a estar geolocalizada en los Estados
Unidos, correspondía a un \emph{host} ubicado físicamente en Argentina. Una
posibilidad es que el acceso al sitio sea brindado a través de una CDN
(Content Delivery Network).

Por este motivo, el experimento se realizó con el dominio \emph{mit.edu}, que,
al momento de llevarlo a cabo, resolvía a la dirección \texttt{IP}
\texttt{104.66.69.243}. Se observó la siguiente ruta:

\begin{figure}[H]
    \figdef[dim]{figures/mit_route_table}
    \caption{Ruta obtenida hacia \emph{mit.edu}.}
    \label{res:escC:table}
\end{figure}

Puede verse que se trata de una ruta considerablemente más corta que las
anteriores, con un total de 10 saltos, de los cuales 4 ($40\%$) no
respondieron con un \emph{Time exceeded}. Cabe destacar que, entre los saltos
que sí respondieron, la única anomalía se observa en el \emph{hop} 5, que
presenta un \texttt{RTT} menor al de los \emph{hops} 6 y 7. Los \texttt{RTT}s
de los demás saltos se encuentran en orden creciente. Esto puede contrastarse
con los experimentos anteriores, donde la proporción de saltos donde se
observan inversiones en el orden de los \texttt{RTT}s es considerablemente
mayor.

En la Figura \ref{res:escC:map} se puede observar el resultado de geolocalizar
las \texttt{IP}s de la ruta mediante GeoIP.

\begin{figure}[H]
    \figdef[dim]{figures/mit_route_map}
    \caption{Localización de saltos según geolocalización de direcciones IP para
    el dominio \emph{mit.edu}.}
    \label{res:escC:map}
\end{figure}

\begin{figure}[H]
    \figdef[dim]{figures/mit_route_rtt_plot}
    \caption{Valores obtenidos para el \texttt{RTT} entre saltos ($X_i$) de la ruta 2B.}
    \label{res:escC:rtt}
\end{figure}


\begin{figure}[H]
    \figdef[dim]{figures/mit_route_norm_rtt_plot}
    \caption{\texttt{RTT} entre saltos normalizado ($\frac{X_i-\bar{X}}{S}$)
    y valor de $\tau_n$ para la ruta 2B.}
    \label{res:escC:rttnorm}
\end{figure}
\section{Conclusiones}

Del trabajo realizado pueden extraerse conclusiones interesantes acerca del
poder de las herramientas utilizadas a la hora de analizar la topología y el
funcionamiento de las rutas en \emph{Internet}.

Es posible notar cómo, para rutas largas, la capacidad de detección de los
saltos intercontinentales parece decaer. Esto es especialmente notorio en el
caso del escenario A, correspondiente a la ruta más extensa de las tres, donde
solo pudo detectarse correctamente uno de tres saltos intercontinentales. Como
ya se discutió en el análisis de esta ruta, esto puede deberse a que las
rutas de tal longitud resultan poco estables. Durante el tiempo que demora
en realizarse el análisis de \emph{traceroute}, es posible que partes de la
ruta se modifiquen, cambiando, por ejemplo, el número de \emph{hop}
correspondiente a un salto intercontinental. De suceder esto, se produciría
un claro efecto negativo en la posibilidad de identificar dicho salto.

Algo para destacar es que, en la mayoría de los casos, el método de detección
de \emph{outliers} utilizando el valor $\tau$ resultó efectivo para
identificar los mismos, al menos partiendo de una inspección visual de los
gráficos obtenidos. Sin embargo, hay una excepción notoria es el caso de la
ruta 2A, donde un valor correspondiente a un salto intercontinental fue
omitido por una diferencia mínima.

Observando los gráficos del \texttt{RTT} por saltos normalizado, es posible
advertir una diferencia clara entre los \emph{hops} con \texttt{RTT}s cercanos
a la media de la distribución, que una vez normalizados no exceden en general
un valor de $0.5$, y los \emph{outliers}, que se encuentran siempre por encima
de $1.5$. Así, un valor de corte fijo en $1.5$ podría haber sido utilizado
para identificar a los \emph{outliers} evitando la omisión recién mencionada.
No obstante, al no tener en cuenta la varianza de la distribución, esta
metodología puede fallar de aparecer rutas donde la diferencia no sea tan
marcada, por lo que su efectividad debería ser probada experimentalmente con
mayor profundidad.

En el paper de Jobst\cite{Jobst} se describen unas cuantas ``anomalías'' de traceroute. En los experimentos realizados aparecieron algunas de ellas, mientras que otras no. Por supuesto, la denominada \emph{missing hops} apareció en todos los experimentos, ya que es sumamente común que algunos \emph{routers} en el camino no respondan. La anomalía denominada \emph{missing destination} también se presentó (es también muy común que un \emph{host} esté configurado para no responder los \emph{ICMP Echo}), pero a la hora de elegir las rutas a analizar, se dejó fuera estos casos. 


También se presentaron problemas en la obtención de los RTT (\emph{False RTTs}). En la ruta C, por ejemplo, se ve un incremento negativo del RTT entre los hops 6 y 7. En la B2, también, entre los hops 16 y 17. Como se discutió previamente en el análisis de la ruta A, esto podría ser indicativo de un path asimétrico entre el envío y la respuesta (toman distintas rutas). No se hallaron situaciones como las descriptas con respecto a los links MPLS.




\bibliographystyle{IEEEbib}
\bibliography{informe}

\end{document}
